% !TEX program = lualatex

%%%%%%%%%%%%%%%%%%%%%%%%%%%%%%%%%%%%%%%%%%%%%%%%%%%%%%%%%%%%%%%%%%%%%
% WARNING: This PDF uses OCG (Optional Content Groups) for toggles.
% SUPPORTED VIEWERS: Okular, Evince, Chrome, Firefox, Acrobat (Desktop).
% NOT SUPPORTED: Zathura, Mobile PDF Viewers, Preview (macOS).
%%%%%%%%%%%%%%%%%%%%%%%%%%%%%%%%%%%%%%%%%%%%%%%%%%%%%%%%%%%%%%%%%%%%%

\documentclass{article}
\usepackage[utf8]{inputenc}
\usepackage{xcolor}
\usepackage{mathabx}
\usepackage{ocgx2}
\usepackage{hyperref}

% Toggle switch macro
\NewDocumentCommand{\itemtoggle}{mm}{%
  {\hypersetup{pdfborder=0 0 0}%
  \switchocg{#2}{%
    \begin{ocg}{#1}{#2}{off}\rlap{$\blacktriangledown$}\end{ocg}%
    \begin{ocmd}{\AllOff{#2}}$\blacktriangleright$\end{ocmd}%
  }}%
}

\begin{document}

% Warning block inside the PDF
\noindent\fcolorbox{red}{yellow!10}{%
	\parbox{\dimexpr\textwidth-2\fboxsep-2\fboxrule\relax}{%
		\textbf{Compatibility Note:} This document contains interactive toggles.
		Please use \textbf{Okular}, \textbf{Evince}, or a \textbf{Desktop Web Browser} to view it.
		Interactive features will not function on mobile devices or Zathura.
	}%
}

\section*{Atmospheric Science \& Physics}

\begin{itemize}
	% Item 1: Hydrostatic Balance
	\item[\itemtoggle{Hydrostatic}{hydro}] \textbf{Hydrostatic Balance}
		\begin{ocmd}{\AllOn{hydro}}
			{\color{teal} $\frac{dp}{dz} = -\rho g$} \\
			Explains why the atmosphere doesn't collapse or fly off into space; pressure decreases with height to balance gravity.
		\end{ocmd}

		% Item 2: Coriolis Force
	\item[\itemtoggle{Coriolis}{cor}] \textbf{Coriolis Parameter}
		\begin{ocmd}{\AllOn{cor}}
			{\color{purple} $f = 2\Omega \sin \phi$} \\
			Crucial for large-scale motion; it accounts for the Earth's rotation and determines the direction of cyclones.
		\end{ocmd}

		% Item 3: Mass-Energy
	\item[\itemtoggle{Relativity}{emc2}] \textbf{Mass-Energy Equivalence}
		\begin{ocmd}{\AllOn{emc2}}
			{\color{cyan} $E=mc^2$} \\
			Einstein's formula relating energy to mass, explaining the nuclear fusion that powers solar radiation.
		\end{ocmd}
\end{itemize}

\end{document}
