\documentclass{article}

\usepackage[
  hmargin = 2.4cm,
  vmargin = 3cm
]{geometry}
\usepackage[
  figureposition = bottom
]{caption}
\usepackage{pst-solides3d}

% Upright text as subscript in math mode.
\makeatletter
 \begingroup
  \catcode`\_=\active
  \protected\gdef_{\@ifnextchar|\subtextup\sb}
 \endgroup
\def\subtextup|#1|{\sb{\textup{#1}}}
\AtBeginDocument{\catcode`\_=12 \mathcode`\_=32768}
\makeatother

% Setup of caption.
\DeclareCaptionLabelSeparator{adjustment}{:\quad}
\captionsetup{
  font = small,
  labelfont = sc,
  labelsep = adjustment,
  width = 0.7\textwidth
}

%% Parameters
% Windings
\def\lWind{40}
\def\rWind{80}
% Radii
\def\rHelix{1.13}
\def\rWire{0.004}

% Constants
\def\factor{160} % \factor > \lWind,\rWind
\pstVerb{%
  /left 2 \lWind\space mul \factor\space div def
  /right 2 \rWind\space mul \factor\space div def
}

%% Colours
\colorlet{wireColor}{red!60}
\colorlet{coreColor}{cyan!50}
%% Wire
\newpsobject{wire}{psSolid}{%
  object = courbe,
  ngrid = 4365 left mul cvi 5,
  r = \rWire,
  fillcolor = wireColor,
  incolor = wireColor
}

\pagestyle{empty}

\begin{document}

\begin{figure}[htbp]
	\centering
	\begin{pspicture}(-6.6,-4.4)(6.6,4.2)
		\psset{%
			algebraic,
			solidmemory,
			viewpoint = 20 5 10 rtp2xyz,
			lightsrc = 20 60 60 rtp2xyz,
			Decran = 30,
			grid = false,
			action = none
		}
		%%--------- Core ----------
		\psSolid[
			object = anneau,
			h = 1.0,
			R = 4,
			r = 2.5,
			ngrid = 4,
			RotX = 90,
			RotY = 45,
			RotZ = 90,
			fillcolor = coreColor,
			name = core
		]
		%%--------- Wire ----------
		% Left
		\defFunction{heliceA}(t){\rHelix*cos(\factor*t)}{\rHelix*sin(\factor*t)}{t/left}
		\wire[
			function = heliceA,
			range = 0 Pi left mul,
			name = wireA
		](0,-2.25,-1.5)
		% Right
		\defFunction{heliceB}(t){\rHelix*cos(\factor*t)}{-\rHelix*sin(\factor*t)}{t/right}
		\wire[
			function = heliceB,
			range = 0 Pi right mul,
			name = wireB
		](0,2.25,-1.5)
		%%------- Assembly --------
		\psSolid[
			object = fusion,
			base = core wireA wireB,
			action = draw**
		]
		%%---- Connecting wire ----
		% Left
		\psline[
			linewidth = 1.5pt
		](-6.8,2.71)(-3.705,2.71)(-3.705,2.31)
		\psline[
			linewidth = 1.5pt
		](-6.8,-2.845)(-3.65,-2.845)(-3.65,-2.545)
		\pcline[
			linewidth = 0.5pt
		]{<->}(-6,2.71)(-6,-2.845)
		\ncput*{\small $U_|p|$}
		\uput[315](-6,2.71){\small $+$}
		\uput[40](-6,-2.845){\small $-$}
		\psline{->}(-6.8,3.01)(-5.5,3.01)
		\uput[0](-5.5,3.01){\small $I_|p|$}
		\rput(-1.3,0){\small $N_|p|$}
		% Right
		\psline[
			linewidth = 1.5pt
		](6.8,2.65)(3.48,2.65)(3.48,2.25)
		\psline[
			linewidth = 1.5pt
		](6.8,-3.0)(3.41,-3)(3.41,-2.7)
		\pcline[
			linewidth = 0.5pt
		]{<->}(6,2.65)(6,-3)
		\ncput*{\small $U_|s|$}
		\uput[225](6,2.65){\small $+$}
		\uput[140](6,-3){\small $-$}
		\psline{->}(5.5,2.95)(6.8,2.95)
		\uput[180](5.5,2.95){\small $I_|s|$}
		\rput(1.3,0){\small $N_|s|$}
	\end{pspicture}
	\caption{Transformer with $\lWind$~windings on the primary side and $\rWind$~windings on the secondary side.}
	\label{fig:transformer}
\end{figure}

\end{document}
