%%%%%%%%%%%%%%%%%%%%%%%%%%%%%%%%%%%%%%%%%%%%%%%%%%%%%%%%%%%%%%%%%%%%%
% WARNING: This PDF uses OCG (Optional Content Groups) for toggles.
% SUPPORTED VIEWERS: Okular, Evince, Chrome, Firefox, Acrobat (Desktop).
%%%%%%%%%%%%%%%%%%%%%%%%%%%%%%%%%%%%%%%%%%%%%%%%%%%%%%%%%%%%%%%%%%%%%

\documentclass{article}
\usepackage[utf8]{inputenc}
\usepackage[T1]{fontenc}
\usepackage{xcolor}
\usepackage{amsmath, amsthm, amssymb}
\usepackage{hyperref}
\usepackage[tikz]{ocgx2}

\usetikzlibrary{calc, shapes.callouts, arrows.meta, shadows}

\hypersetup{
    colorlinks=true,
    linkcolor=blue,
    pdfborder={0 0 0}
}

% Helper for clickable text anchors
\newcommand{\jymanchor}[1]{\tikz[remember picture,baseline=(#1.base)]{\node(#1){};}}

\begin{document}

% Warning block inside the PDF
\noindent\fcolorbox{red}{yellow!10}{%
	\parbox{\dimexpr\textwidth-2\fboxsep-2\fboxrule\relax}{%
		\textbf{Compatibility Note:} This document contains interactive toggles.
		Please use \textbf{Okular}, \textbf{Evince}, or a \textbf{Desktop Web Browser} to view it.
		Interactive features will not function on mobile devices or Zathura.
	}%
}

\section*{Atmospheric Science \& Physics}

\begin{itemize}
	% Trigger 1: Hydrostatic (Dimming)
	\item \textbf{Vertical Dynamics:}
	      \switchocg{hydro_text}{%
		      \textcolor{teal}{\textbf{Hydrostatic Balance\jymanchor{hydro_text_anchor}}}%
	      } (Dimming style)

	      % Trigger 2: Coriolis (Simple)
	\item \textbf{Style 2 (Simple):}
	      \switchocg{cor_details}{%
		      \textcolor{purple}{\textbf{Coriolis Effect\jymanchor{cor_link}}}%
	      } (Floating style)

	      % Trigger 3: Hydrostatic (Diagram)
	\item \textbf{Style 3 (Diagram):}
	      \hideocg{geo_diag}{%  <-- Auto-closes Geostrophic if open
		      \switchocg{hydro_diag}{%
			      \textcolor{teal}{\textbf{Hydrostatic Balance\jymanchor{hydro_diag_anchor}}}%
		      }%
	      } (Diagram style)

	      % Trigger 4: Geostrophic Wind (Mutually Exclusive with Style 3)
	\item \textbf{Style 4 (Dynamics):}
	      \hideocg{hydro_diag}{% <-- Auto-closes Hydrostatic if open
		      \switchocg{geo_diag}{%
			      \textcolor{blue}{\textbf{Geostrophic Wind\jymanchor{geo_link}}}%
		      }%
	      } (Mutually exclusive with Style 3)
\end{itemize}

% --- Layer 1: Hydrostatic Text (Dimming) ---
\begin{tikzpicture}[remember picture, overlay]
	\begin{ocg}{Hydrostatic Text}{hydro_text}{off}
		\fill[white, opacity=0.8] (current page.south west) rectangle (current page.north east);
		\node[draw=teal, fill=teal!5, thick, rectangle callout, rounded corners,
			callout absolute pointer=(hydro_text_anchor), text width=6cm, inner sep=12pt, drop shadow]
		at ($(hydro_text_anchor)+(4,2)$) {
			\textbf{Hydrostatic Equation}\\
			\[ \frac{\partial p}{\partial z} = -\rho g \]
		};
	\end{ocg}
\end{tikzpicture}

% --- Layer 2: Coriolis (Simple Pop-up) ---
\begin{tikzpicture}[remember picture, overlay]
	\begin{ocg}{Coriolis}{cor_details}{off}
		\node[draw=purple, fill=purple!5, rectangle callout, rounded corners,
			callout absolute pointer=(cor_link), text width=5cm]
		at ($(cor_link)+(3,-2)$) {
			\textbf{Coriolis Force}\\
			Deflects motion to the right in the Northern Hemisphere.
		};
	\end{ocg}
\end{tikzpicture}

% --- Layer 3: Hydrostatic Diagram ---
\begin{tikzpicture}[remember picture, overlay]
	\begin{ocg}{Hydrostatic Diagram}{hydro_diag}{off}
		\fill[black, opacity=0.1] (current page.south west) rectangle (current page.north east);
		\node[draw=teal, fill=white, thick, rounded corners,
			text width=7.5cm, inner sep=15pt, drop shadow]
		at ($(current page.center)+(0,3)$) {
			\textbf{Hydrostatic Balance Diagram}\\
			\vspace{0.3cm}
			\begin{tikzpicture}[scale=0.9]
				\draw[thick, fill=teal!10] (0,0) rectangle (3,1.5);
				\draw[-{Latex}, red, ultra thick] (1.5,1.5) -- (1.5,2.5) node[right] {$P_{bottom}$};
				\draw[-{Latex}, blue, ultra thick] (1.5,0) -- (1.5,-1) node[right] {$P_{top} + g$};
				\node at (1.5,-1.8) {\large $\frac{\partial p}{\partial z} = -\rho g$};
			\end{tikzpicture}
		};
	\end{ocg}
\end{tikzpicture}

% --- Layer 4: Geostrophic Diagram ---
\begin{tikzpicture}[remember picture, overlay]
	\begin{ocg}{Geostrophic Diagram}{geo_diag}{off}
		\fill[black, opacity=0.1] (current page.south west) rectangle (current page.north east);
		\node[draw=blue, fill=white, thick, rounded corners,
			text width=7.5cm, inner sep=15pt, drop shadow]
		at ($(current page.center)+(0,3)$) {
			\textbf{Geostrophic Balance}\\
			\vspace{0.3cm}

			\begin{tikzpicture}[scale=0.8]
				\draw[dashed, gray] (0,0) -- (5,0) node[right] {High $P$};
				\draw[dashed, gray] (0,2) -- (5,2) node[right] {Low $P$};
				% Vectors
				\draw[-{Stealth}, green!60!black, ultra thick] (2.5,1) -- (4.5,1) node[right] {$V_g$};
				\draw[-{Latex}, red, thick] (2.5,1) -- (2.5,2.5) node[above] {PGF};
				\draw[-{Latex}, blue, thick] (2.5,1) -- (2.5,-0.5) node[below] {$F_{cor}$};
			\end{tikzpicture}
			\small Wind flows parallel to isobars.
		};
	\end{ocg}
\end{tikzpicture}

\end{document}
